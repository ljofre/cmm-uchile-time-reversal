\section{Descripción del Software y Manual de Uso}

\subsection{Forma de importar los datos al sistema}
La información desde los archivos de texto codificados en ascii se importan por
medio de un script en python. El script toma cada archivo de los eventos
sísmicos y genera una carpeta con un nombre y además con el sufijo FOLDER.
 
\subsection{Descripción de estructuras de datos}
Las información de un evento está almacenada en dos clases matlab con los
nombres Event.m y Geonsensor.m los cuales están relacionados como se muestra
en la figura.



El detalle de estos clases se muestra en la siguiente lista:

Los métodos de las clases se describirán en el resumen de la implementación
de las funciones en matlab.

\subsection{Resumen de implementacion (En Matlab y Python)}

\subsubsection{Scripts en Python}
\paragraph{readFile.py}
Se usa una rutina en python para separar la informacón desde los documentos
de los geófonos y convertirla a archivos que servirán de imput que funciones
matlab trasnformen la información a objetos matlab.
\paragraph{patterns.py}
Conjunto de expresiones regulares con la información necesaria que luego se
extraerá mediante readFile.py.

\subsubsection{Clases Matlab}
\paragraph{Event.m}
Clase que contiene una lista de geófonos y los parámetros de un evento
sísmico especifico. Estos son los atributos de los set de datos entregados
por Codelco. Los atributos principales son de esta clase son:
\begin{itemize}
  \item count: Cantidad de geofonos que hicieron mediciones en el evento sísmico
  medido.
  \item name: Nombre del evento sísmico y que tiene caracter único.
  \item alpha: Velocidad de la onda $p$.
  \item beta: Velocidad de la onda $s$.
  \item rho: Densidad del medio rocoso
  \item LocR: Ubicación estimada de la fuente dada por el set de datos.
  \item origintime: Tiempo estimado de la fuente dado por el set de datos.
  \item gss: lista de objetos del tipo \textbf{geosensor} que representan a cada
  uno de los geófonos con sus atributos y mediciones.
  \item src: Donde se almacena la estimación de la fuente
  \item filtsrc: Estimación de la fuente filtrada.
  \item err: Error de estimación de la fuente por mínimos cuadrados.
\end{itemize}

\paragraph{Geonsensor.m}
Clase que contiene la información de cada uno de los geofonos en un evento
específico. Contiene las mediciones y atributos que definen totalmente a un
geofono en un evento en específico.

\begin{itemize}
    \item firsttime: Tiempo de la primera medición del geófono
    \item lasttime: Tiempo de la última medición del geófono
    \item resampleSize: Cantidad de mediciones del campo de desplazamiento del
    sismógrafo remuestreado.
    \item timeresamplevector: Vector de tiempo remuestreado.
    \item timevector: Vector de tiempo de la medición
    \item diferenciaPSvalida: es 0 si el tiempo de llegada de la onda $p$ es
    menor al tiempo de llegada de la onda $s$
    \item medicionesValidas: Vector Booleano $(b_1,b_2,b_3)$ que dice que
    dimensiones del sismográma tiene mediciones.
    \item r0: Posición física del geófono.
    \item hardware\_sampling\_rate: Frecuencia de muestreo del geófono.
    \item resampling\_rate: Frecuencia variable de remuestreo del campo de
    desplazamiento de la señal.
    \item period: Periodo variable, cumple la relación period = 1/resampling\_rate
    \item TriggerPosition: indice del tiempo estimado en donde ocurrió el evento
    sísmico.
    \item r\_x: Campo de desplazamiento remuestreado en el eje $x$
    \item r\_y: Campo de desplazamiento remuestreado en el eje $y$
    \item r\_z: Campo de desplazamiento remuestreado en el eje $z$
    \item L: Cantidad de mediciones del sismógrama.
    \item data: Mediciones.
    \item t\_time: Tiemoo en el cual
    \item p\_time: Tiempo de llevada de la onda $p$
    \item s\_time: Tiempo de llegada de la onda $s$
    \item validP: Flag que indica la validez de la onda $p$
    \item validS: Flag que indice la validez de la onda $s$
    \item validSP Flag que indica la validez de ambas ondas no, necesariamente
    es la conjunción de validP y validS. 
    \item IsAccelerometer: Flag que indica si es el geófono es un acelerómetro. 
    \item IsSpeedometer: Flag que indica si el geófono es un velocímetro.
    \item sensor\_id: Clave identificadora del geófono.
    \item validAll: Flag que dice si en cada uno de los ejes existen mediciones
    válidas.
\end{itemize}

\subsubsection{Documentación de las Rutinas Principales}

\paragraph{Importar información a matlab desde los archivos procesados}
\begin{itemize}
  \item Archivo: script/importEvents.m
  \item Comando: events = importEvents()
  \item Descripción: Almacena la información desde los archivos tratados por
  el script python readFile.py a una vector de objetos Event en matlab.
  \item input: no recibe parámetros de entrada dado que hace lecturas sobre
  archivos de texto almacenados en disco.
  \item output: event lista de objetos del tipo Event que contienen a todos los
  eventos sísmicos almacenados en la carpeta `./project/data sets`  		 
\end{itemize}


\paragraph{Estimar la fuente de un evento sísmico como una fuerza}
\begin{itemize}
  \item Archivo: script/source.m
  \item Dependencia:   filterLowPassSersor.m scalarGreenKernel.m
  \item Comando: [src, filtsrc, error] = source(event, nSrc, L, por)
  \item Dependencia:  filterLowPassSersor.m scalarGreenKernel.m
 
  \item Descripción: Estima la fuente como una fuerza $\hat{s}(t)$ en un intervalo de 
  tiempo de largo $L$ dado el conjunto de sismógrafos que definen al evento por medio
  de mínimos cuadrados. Los geofonos son los mismos de  \textbf{event} con una 
  discretización del dominio con \textbf{nSrc} con una
  fracción \textbf{por} de la fuente antes de \textbf{event.origin\_time} es
  cual es el valor estimado por codelco.
  \item input: 
    \begin{itemize}
    \item event: Objeto del tipo Event el cual representa el sísmico del cual se
    quiere obtener la estimación de la fuente $f(t)$ como una fuerza.
    \item nSrc: número de puntos de la discretización uniforme en el tiempo con
    la cual se va a estimar la fuente.
    \item L: largo de la ventana de tiempo en donde se va a estimar la fuente
    \item por: porcentaje de la ventana de tiempo que está antes del tiempo de
    origen estimado por el documento del evento.
    \end{itemize}
  \item output:
  	\begin{itemize}
  	  \item src: fuente sismica como una fuerza en el punto $r_0$ estimado desde
  	  los valores dados por las mediciones de los geofonos
  	  \item filtsrc: src pero filtrada eliminando los modos bajos
  	  \item error: error de estimación $\sum_{k,t} \|\hat{u}_k[t]
  	  - u_k[t]\|_2^2$
  	\end{itemize}
\end{itemize}

\paragraph{construcción de un sismograma dada una fuente}
\begin{itemize}
  \item Archivo: script/constructsensor.m
  \item Dependencia: scalarGreenKernel.m
  \item Comando: [gsRec] = constructsensor(event, index, src)
  \item Descripción: de el evento \textbf{event} considera el dominio temporal
  del geófono número \textbf{index} y recosntruye la señal en ese dominio dado
  por una fuente de la forma \textbf{src}.
  \item input:
  \begin{itemize}
    \item event: Evento del cual se desea hacer una reconstrucción de los 
    sismogramas
    \item index: Indice del geófono que se quiere estimas dentro de la lista 
    existente en \textbf{event}
    \item src: Fuente sísmica como fuerza que al ser convolucionada con la 
    función de Green de la ecuación de onda elástica recontruirá al sismograma 
    del geófono index.
  \end{itemize}
  \item output:
  \begin{itemize}
    \item gsRec: sismograma reconstruido.
  \end{itemize}
\end{itemize}

\paragraph{Función de Green}
\begin{itemize}
  \item Archivo: script/scalarGreenKernel.m
  \item Comando: [G11,G12,G13,G22,G23,G33] = scalarGreenKernel(x,y,z,T, alpha,
  beta,rho)
  \item Descripción: retorna cada uno de los elementos de la matriz simétrica
  que determina a la función de Green de forma discreta.
  \item input:
  \begin{itemize}
    \item T: Dominio temporal discretizado en donde se va a calcular la función
    de Green.
    \item (x,y,z): traslación $r$ para hacer el cálculo de $G(r-r_0,t-t_0)$
    \item t: traslación de $t$ para hacer el cálculo de $G(r-r_0,t-t_0)$
    \item alpha, beta, rho: velocidad de la onda $p$, onda $s$ y densidad del 
    medio respectivamente.
  \end{itemize}
  \item output:
  \begin{itemize}
    \item G11,G12,G13,G22,G23,G33 : Componentes de la matriz simétrica que
    representa a la función de Green para la ecuación diferencial elástica.
    \item 
  \end{itemize}
\end{itemize}

\paragraph{Inversión de un conjunto de sismogramas en el tiempo}
\begin{itemize}
	\item Archivo: sripts/reverse\_signal.m
	\item Dependencia: field.m
	\item Comando: [X, Y, Z, X\_domain, Y\_domain, Z\_domain, T\_domain] =
	reverse\_signal(obj)
	\item Descripción:Genera la inversión en el tiempo de los sismogramas, eso   
	se obtiene considerando al campo de desplazamiento del sismograma con una 
	fuente.
	\item input:
\end{itemize}

\paragraph{Rotación de una fuente sísmica}
\begin{itemize}
  \item Archivo: scripts/rotate.m
  \item Comando: [rotatesrc] = rotate(src)
  \item Descripción: Toma un evento sísmico y lo rota mediante un cambio de base
  con respecto los vectores propios de la matriz de covariaza entre los ejes de
  la estimación de la fuente como una fuerza.
  \item src: Fuente sismica
  \item rotatesrc: Fuente sismica rotada
\end{itemize}


\paragraph{Remuestreo de un campo de desplazamiento de un sismograma}
\begin{itemize}
  \item Archivo: scripts/variableResample.m
  \item Comando: geo = variableResample(geo, error)
  \item Dependencia: find\_tail\_limit.m
  \item Descripción: Toma el campo de desplazamiento y la remuestre de forma
  variable de tal manera de almacenar mediante una cantidad mínima de datos la
  señal completa. La cantidad de datos remuestreados son los suficientes que
  produzcan que $\|\hat{u} - u\|_1<=\text{error}$
  \item input:
  \begin{itemize}
    \item geo: Objeto del tipo geosensor sin las mediciones remuestreadas
    \item error: Hace referencial al error de remuestreo
  \end{itemize}
  \item out: geo: Objeto del tipo geosensor con dados remuestreados almacenados
  en los atributos r\_x, r\_y, r\_z.
\end{itemize}

\paragraph{Seleccionar la parte de la señal de un sismograma}
\begin{itemize}
  \item Descripción: El costo de encontrar una fuente sismica depende de la
  cantidad de mediciones implicadas en cada uno de los sismogramas, como una
  parte del sismograma son solo mediciones ruidosas con se puede omitir para
  hacer una estimación más rápida en terminos computacionales.
\end{itemize}
\paragraph{Clusterización de los eventos sismicos}
 \begin{itemize}
   \item Descripción: Visualización de la agrupación de las mediciones por medio
   de un algoritmos de agrupación de k-medias dada una norma $l_2$ de las
   componentes filtradas y rotadas de la fuente sísmica (fuerza).
 \end{itemize}

\subsection{Manual de uso}
La importación de los datos desde los archivos dados se hacen mediante el
program en python readFile.py que fué programado y testeado en un ambiente
linux, dado que las rutas a carpetas son distintas entre los distintos sistemas
operativos como lo son windows y otros basados en UNIX se hace necesaria la
advertencia.
El uso del programa propiamente tal está descrito en dos archivos archivos
matlab, en el primero sampletimereversal.m se describe como invertir la señal en
el tiempo. En el segundo archivo sampleRecSource.m se describen los pasos para
la reconstruccion de una fuente sismica a partir de la información de los
geofonos dados por Codelco.






















